\documentclass{ctexart}

\usepackage{amsmath, amssymb}
\usepackage{array, tabularx}
\usepackage{graphicx}
\begin{document}



“你好,世界!”来自 \LaTeX{} 的问候。\\
\title{l}
\author{me}
\date{\today}
\maketitle
\begin{abstract}
    摘要在这里,一般在一页开头。

\end{abstract}
\tableofcontents 

“你好,世界!”来自 \LaTeX{} 的问候。\\
11111
 \\ \par

\section{中文}

在\LaTeX{}中排版中文。
汉字和English单词混排,通常不需要在中英文之间添加额外的空格。
当然,为了代码的可读性,加上汉字和 English 之间的空格也无妨。
汉字换行时不会引入多余的空格。\\
You know
I am learning \LaTeX{}.\\
\LaTeX{} \\
\# \$ \& \% \_ \{ \} \~{} \^{}  \textbackslash \\
It's difficult to find \ldots\newline
It's dif{}f{}icult to f{}ind \ldots \\
%特殊的符号
``double quote''-`single quote'--2--3---\\

\dots \\
\ldots \\
Donald~E.~Knuth \newline
111 

\newpage
222 \par

A reference to this subsection
\label{aaa} looks like:
``see section~\ref{aaa} on
page~\pageref{aaa}.''
\footnote{这是一个脚注}
\marginpar{\footnotesize 小字边注}


\renewcommand{\labelenumi}%
{\Alph{enumi}>}
\renewcommand{\labelitemi}{\ddag}
\begin{enumerate}
\item 1111
\item 2222
\begin{itemize}
    \item 3333
    \item 4444
\end{itemize}
\item 5555
\end{enumerate}

\begin{description}
\item[Main] a
\item[Test] b
\item[Eval] c  
\end{description}

\begin{center}
    中心对齐的文本
\end{center}

\begin{flushleft}
    左对齐的文本
\end{flushleft}

\begin{flushright}
    右对齐的文本
\end{flushright}

\centering 中心对齐的文本
1111111 \\
\raggedleft
右对齐的文本 \\
\raggedright
2222222 \par
短文本:
\begin{quote}
    引用较短的文字
\end{quote}
长文本:
\begin{quotation}
引用几段文字或者长文字

引用几段文字或者长文字
\end{quotation}
诗歌:
\begin{verse}
    引用诗歌

    首行悬挂缩进

\end{verse}

\begin{center}
    $ \min(Y-\phi\widehat{\theta})^{T}(\theta-\phi\widehat{\theta}) $\\
    $ \widehat{\theta}=(\phi^{T}\phi)^{-1}\phi^{T}Y   $\\
    $ A_i=\int_0^\infty [1-h^{*}(t)]\frac{(-t)^{i-1}}{(i-1)!}dt+\sum_{k=1}^{i-1} A_{i-k} \int_0^\infty [1-h^{*}(t)]\frac{(-t)^{i-1}}{(i-1)!} dt$

\end{center}




% 正文中的使用
\begin{equation*}
\begin{bmatrix}a_1 \\ a_2 \\\vdots \\ a_n\end{bmatrix}=\begin{bmatrix}A_1 \\ A_2 \\\vdots \\ A_n\end{bmatrix}
\end{equation*}

$ K_{N+1}=P_N\phi_{N+1}(1+\phi_{N+1}^{T}P_N\phi_{N+1})^{-1} $\\
$ \widehat{\theta_{N+1}}=\widehat{\theta_N}+K_{N+1}(y_{N+1}-\phi_{N+1}^T\widehat{\theta_N}) $\\
$ P_{N+1}=P_N-K_{N+1} \cdot \phi_{N+1}^T \cdot P_N $\\


$ I=\frac{R+G+B}{3} $\\
$ S=1-\frac{3}{R+G+B}[min(R,G,B)] $\\
$ H=\arccos\frac{[(R-G)+(R-B)]/2}{[(R-G)^2+(G-B)(R-B)]^\frac{1}{2}} $




\begin{verbatim}
#include <iostream>
int main()
{
    std::cout << "Hello, world!"
        << std::endl;
    return 0;
}
\end{verbatim}
\verb|(a || b)++|
\\
\centering
\begin{tabular}{|c|c|c|}
  \hline
  左对齐 & 居中 & 右对齐 \\
  \hline
  数据1 & 数据2 & 数据3 \\
  \cline{1-2}
  数据4 & 数据5 & 数据6 \\
  \hline
\end{tabular}
\par
\begin{tabularx}{14em}
  {|*{4}{>{\centering\arraybackslash}X|}}
  \hline
  A & B & C & D \\
  \hline
  ABC & BCD & CDE & DEF \\
  \hline
\end{tabularx}
\par
\begin{tabular}{|c|c|c|}
    \hline
    1 & 2 & 3 \\
    \hline
    \multicolumn{2}{|c|}{3} & 4\\
    \hline
    
\end{tabular}
\par
\includegraphics[scale = 0.1]{test.jpg}
\par
\makebox[10em][c]{Test Some Words} \\
\framebox[10em][c]{Test Some Words} \\
\raggedright
\begin{minipage}{3em}
    数据1 数据2 数据3
\end{minipage}
\par
\rule{5pt}{8pt}\\

\begin{table}[h]

  这是一个浮动体\caption{标题}

\end{table}

\par 
The Pythagorean theorem is
$a^2 + b^2 = c^2$.

\begin{equation}
    a^2 + b^2 = c^2
\end{equation} \\
\[ a^2 + b^2 = c^2 \] \par
中文之间的空格会被忽略,使用\textbackslash quad可以 \quad 解决 \\

$x^{2} \geq 0 \qquad
\text{for \textbf{all} }
x$ \\
$f(x) = x^2 \quad f'(x) = 2x \quad f''^{2}(x) = 4$ \\
$ \sqrt[3]{x} \quad \sqrt[4]{x} \quad \sqrt[5]{x} \quad \sqrt[6]{x} $\\
$ \binom{n}{k} = \binom{n}{k-1} + \binom{n-1}{k-1} $\\
$ \ne \ge \le \approx \equiv \propto \sim $ \\
$ \overbrace{(a+b+c)}^6 \cdot \underbrace{(a+b+c)}_7 = 42 $ \\


\begin{multline}
a + b + c + d + e + f + g + h + i \\
= j + k + l + m + n\\
= o + p + q + r + s\\
= t + u + v + x + z
\end{multline}
\end{document}

