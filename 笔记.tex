\documentclass{ctexart}

\usepackage{array, tabularx}

\begin{document}
%  \\只用于分行(可选择增加竖直间距)
% \newline :强制换行
% \par :段落结束,开启新段落

% \newpage:强制分页
% \clearpage:强制分页并清除浮动对象

% 断词符号 \-

% 断行或断页的适合度 \linebreak[n] \pagebreak[n] \nolinebreak[n] \nopagebreak[n]

% \input{文件名}:将指定文件的内容插入到当前位置

% 空格的处理 
%英文之间直接输入空格
%中文和英文之间会自动处理空格
%中文之间的空格会被忽略
%不会断行的空格:~
%特殊字符
% \# \$ \& \% \_ \{ \} \~{} \^{}  \textbackslash 

% 英文连字  {} 用于阻止连字

%引号  `'  ``''

%横线 - 连字符 -- 短破折号 --- 长破折号

%省略号 \ldots \dots

%• article 文档类带编号的层级为 \section、\subsection、\subsubsection 三级;
%• report 和 book 文档类带编号的层级为 \chapter、\section、\subsection 三级

% 生成目录: \tableofcontents

% 标题页: \title{title} \author{author} \date{date} \maketitle

% 交叉引用: \label{label} \ref{label} \pageref{label}
% \label{} 位置: 

%脚注 \footnote{脚注内容}
%边注 \marginpar{\footnotesize 边注内容}   字数更小

%列表 有序列表 列表可以嵌套使用,最多嵌套四层
\begin{enumerate}
\item[⟨item title⟩] …
\end{enumerate}
%有序列表的符号由命令 \labelenumi 到 \labelenumiv 定义
% \renewcommand{\labelenumi}%
% {\Alph{enumi}>}


%无序列表
\begin{itemize}
\item[⟨item title⟩] …
\end{itemize}
%符号由命令 \labelitemi 到 \labelitemiv 定义
%\renewcommand{\labelitemi}{\ddag}
%\renewcommand{\labelitemii}{\ddag}

%描述列表
\begin{description}
\item[⟨item title⟩] …
\end{description}

%对齐环境
\begin{flushleft}
\end{flushleft}

\begin{flushright}
\end{flushright}

\begin{center}
\end{center}
%改变对齐方式,使用后要调整回来
%\centering
%\raggedleft
%\raggedright

%引用环境
\begin{quote}
\end{quote}

\begin{quotation}
\end{quotation}

\begin{verse}
\end{verse}

%摘要环境
\begin{abstract}
\end{abstract}


%代码环境
% 回车和空格也分别起到换行和空位的作用
\begin{verbatim}
#include <iostream>
int main()
{
    std::cout << "Hello, world!"
    << std::endl;
    return 0;
}
\end{verbatim}
%\verb 命令   排版简短的代码或关键字
\verb⟨delim⟩⟨code⟩⟨delim⟩
%⟨delim⟩习惯上使用 | 符号

%表格 tabular
%列格式:
% |c| 居中
% |l| 左对齐
% |r| 右对齐
% p{⟨width⟩} 自定义宽度
\begin{tabular}{⟨列格式⟩ p{⟨width⟩}}
  ⟨表格内容⟩
\end{tabular}
%高级使用表格 array宏包 tabularx宏包
%X 列格式 自动计算列宽
\begin{tabularx}{14em}
  {|*{4}{>{\centering\arraybackslash}X|}}
  \hline
  A & B & C & D \\
  \hline
  ABC & BCD & CDE & DEF \\
  \hline
\end{tabularx}
%\cline{⟨i⟩-⟨j⟩} 用来绘制跨越部分单元格的横线
%合并单元格
%\multicolumn{⟨n⟩}{⟨column-spec⟩}{⟨item⟩}

%图片 
% 需要调用graphicx宏包
%\includegraphics[⟨options⟩]{⟨filename⟩}
%option选项
% width=⟨length⟩ 设置图片宽度
% height=⟨length⟩ 设置图片高度
% scale=⟨factor⟩ 缩放图片
% angle=⟨angle⟩ 旋转图片

% 盒子
%水平盒子
%\makebox[⟨width⟩][⟨align⟩]{…}

%带框的盒子
%\framebox[⟨width⟩][⟨align⟩]{…}

%可以换行的盒子
%\parbox[⟨align⟩][⟨height⟩][⟨inner-align⟩]{⟨width⟩}{…}
\begin{minipage}[⟨align⟩][⟨height⟩][⟨inner-align⟩]{⟨width⟩}
…
\end{minipage}

%实心盒子
%\rule[⟨raise⟩]{⟨width⟩}{⟨height⟩}

% 浮动体
\begin{table}[⟨placement⟩]
  
\end{table}
\end{document}