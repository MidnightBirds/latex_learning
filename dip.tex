\documentclass{ctexart}
\usepackage[landscape, margin=0.5cm]{geometry}
\pagestyle{empty}
\setlength{\parindent}{0pt}
\begin{document}
\fontsize{8.5pt}{10pt}\selectfont
\noindent
\begin{minipage}[t]{0.315\linewidth}
\vspace{0pt}
\textbf{第一章绪论:}\\
1.像素在数字图像上具有固定的(x,y)空域坐标以及该点的值f(x,y),其特征为:
x,y,f(x,y)均为离散值和有限值。(f(x,y)常代表图像的intensity或gray等级)
(空域坐标系下y右x)\\
2.数字图像的形式:单通道(B\&W或者灰度等级)\ 三通道(RGB)\ 四通道(RGB+Alpha)深度\\
3. \textbf{(光杆颜锥)}锥状体(少)集中在中央凹附近,对颜色极其敏感;杆状体(多)的分布较为分散,
对低亮度的光照敏感。(两者均不在盲点上有分布)(锥状体(cones),杆状体(rods))\\
4.\textbf{图像不仅仅由电磁波生成,超声波等都可以}(B超)\\
5.\textbf{DIP与计算机视觉、机器人视觉、机器视觉的区别:}\textbf{DIP:}对图像的基础
操作,如几何变换,增强恢复,锐化滤波等,结果仍为图像。\textbf{计算机视觉:}对给定的图
像进行信息提取,如运动检测物体识别,获得具体数据。\textbf{机器视觉:}用于自动化生产,
由机器自主获取图像等信息来获得具体数据,需要多传感器结合。\textbf{机器人视觉:}对象是
机器人,使机器人具备视觉能力完成各项任务。\\
\textbf{第二章数字图像基础:}\\
1.采样量化:数字图像的生成需要经过采样和量化的过程。坐标值(x,y)的数字化就是采样,
值f(x,y)的数字化就是量化。\\
2.\textbf{灰度分辨率:}$2^k$.大小为$M\times N$的数字图像一般用矩阵表示,其占用的储存空间
为:$M\times N\times $k(比特)。空域分辨率为$M\times N$。\\
3.图片的Zooming和Shrinking可用最邻近插值或者双线性插值,前者的缺点是精度低,可能会存在灰度
上的不连续,在变化的地方出现明显的锯齿状。
双线性:$g\left(E\right)=\left(x^\prime-i\right)\left[g\left(B\right)-g\left(A\right)\right]+g\left(A\right)\ $
\\$g\left(F\right)=\left(x^\prime-i\right)\left[g\left(D\right)-g\left(C\right)\right]+g\left(C\right)\ $
\\$g\left(E\right)=\left(x^\prime,y^\prime\right)\left[g\left(F\right)-g\left(E\right)\right]+g(E) $
\\(无论上、降采样,图片所包含内容不变,只是空间分辨率变了)\\
4.像素p的4邻域(十字形$N_4$)、D邻域(4个对角D)和8邻域(一圈$N_8$)。
\textbf{两个像素p和q之间的4邻接(p在q的$N_4$)、8邻接(p在q的$N_8$)和m邻接
(p在q的$N_D$且p的$N_4$与q的$N_4$相交为空;或p在q的$N_4$)}\\
连通(所有连通的像素构成通路):邻接+像素值都属于集合V。
连通分量: 对于S中的任何像素p,S中连通到该像素的像素集
称为S的连通分量。连通集:只有一个连通分量的集合。\\
5.欧几里得距离(De)、街区距离(D4 $\left|x-s\right|+\left|y-t\right|$)和棋盘距离
(D8 max($\left|x-s\right|,\left|y-t\right|$))的概念。(注:D4为菱形,D8为方形)\\
\textbf{第三章:灰度变换和空间滤波(图像增强)高光\ 降噪\ 视觉感染力}\\
1.对于空域,我们的操作一般是针对像素的邻居。若直接对像素本身进行操作,则称为灰度
变换s=T(r);否则为空间滤波g(x,y)=T(f(x,y))\ ,多使用掩模。\\
2.aaaaaaaaaaaaaaaaaaaaaaaaaaaaaaaaaaaaaaaaaa\\
aaaaaaaaaaaaaaaaaaaaaaaaaaaaaaaaaaaaaaaaaa\\


\end{minipage}%
\hfill
\begin{minipage}[t]{0.315\linewidth}
\vspace{0pt}
这里是第二列的内容。文字会正常换行,这里可以写很多文字来测试换行效果。
\end{minipage}%
\hfill
\begin{minipage}[t]{0.315\linewidth}
\vspace{0pt}
这里是第三列的内容。文字会正常换行,这里可以写很多文字来测试换行效果。
\end{minipage}

\end{document}