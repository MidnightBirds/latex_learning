\documentclass{ctexart}
\usepackage[landscape, margin=0.5cm]{geometry}
\pagestyle{empty}
\setlength{\parindent}{0pt}
\begin{document}
\fontsize{8.5pt}{10pt}\selectfont
\noindent
\begin{minipage}[t]{0.315\linewidth}
\textbf{第一章绪论:}\\
1.像素在数字图像上具有固定的(x,y)空域坐标以及该点的值f(x,y),其特征为:
x,y,f(x,y)均为离散值和有限值。(f(x,y)常代表图像的intensity或gray等级)
(空域坐标系下y右x)\\
2.数字图像的形式:单通道(B\&W或者灰度等级)\ 三通道(RGB)\ 四通道(RGB+Alpha)深度\\
3. \textbf{(光杆颜锥)}锥状体(少)集中在中央凹附近,对颜色极其敏感;杆状体(多)的分布较为分散,
对低亮度的光照敏感。(两者均不在盲点上有分布)(锥状体(cones),杆状体(rods))\\
4.\textbf{图像不仅仅由电磁波生成,超声波等都可以}(B超)\\
5.\textbf{DIP与计算机视觉、机器人视觉、机器视觉的区别:}\textbf{DIP:}对图像的基础
操作,如几何变换,增强恢复,锐化滤波等,结果仍为图像。\textbf{计算机视觉:}对给定的图
像进行信息提取,如运动检测物体识别,获得具体数据。\textbf{机器视觉:}用于自动化生产,
由机器自主获取图像等信息来获得具体数据,需要多传感器结合。\textbf{机器人视觉:}对象是
机器人,使机器人具备视觉能力完成各项任务。\\
\textbf{第二章数字图像基础:}\\
1.采样量化:数字图像的生成需要经过采样和量化的过程。坐标值(x,y)的数字化就是采样,
值f(x,y)的数字化就是量化。\\
2.\textbf{灰度分辨率:}$2^k$.大小为$M\times N$的数字图像一般用矩阵表示,其占用的储存空间
为:$M\times N\times $k(比特)。空域分辨率为$M\times N$。\\
3.图片的Zooming和Shrinking可用最邻近插值或者双线性插值,前者的缺点是精度低,可能会存在灰度
上的不连续,在变化的地方出现明显的锯齿状。
双线性:$g\left(E\right)=\left(x^\prime-i\right)\left[g\left(B\right)-g\left(A\right)\right]+g\left(A\right)\ $
\\$g\left(F\right)=\left(x^\prime-i\right)\left[g\left(D\right)-g\left(C\right)\right]+g\left(C\right)\ $
\\$g\left(E\right)=\left(x^\prime,y^\prime\right)\left[g\left(F\right)-g\left(E\right)\right]+g(E) $
\\(无论上、降采样,图片所包含内容不变,只是空间分辨率变了)\\
4.像素p的4邻域(十字形$N_4$)、D邻域(4个对角D)和8邻域(一圈$N_8$)。
\textbf{两个像素p和q之间的4邻接(p在q的$N_4$)、8邻接(p在q的$N_8$)和m邻接
(p在q的$N_D$且p的$N_4$与q的$N_4$相交为空;或p在q的$N_4$)}\\
连通(所有连通的像素构成通路):邻接+像素值都属于集合V。
连通分量: 对于S中的任何像素p,S中连通到该像素的像素集
称为S的连通分量。连通集:只有一个连通分量的集合。\\
5.欧几里得距离(De)、街区距离(D4 $\left|x-s\right|+\left|y-t\right|$)和棋盘距离
(D8 max($\left|x-s\right|,\left|y-t\right|$))的概念。(注:D4为菱形,D8为方形)\\
\textbf{第三章:灰度变换和空间滤波(图像增强)高光\ 降噪\ 视觉感染力}\\
1.对于空域,我们的操作一般是针对像素的邻居。若直接对像素本身进行操作,则称为灰度
变换s=T(r)。否则为空间滤波g(x,y)=T(f(x,y))\ ,多使用掩模。\\
2.如果想将一个物体从背景分离,可以使用阈值变换
$s=1.0(r>threshold)+0.0(r<=threshold)$。若输入的图像灰度分辨率很大,可使用
\textbf{对数变换}$s=c\ast log(l+r)$。
\end{minipage}%
\hfill
\begin{minipage}[t]{0.315\linewidth}
\textbf{幂律变化}$s=c\ast r^y$则可以将一个较窄范围的灰度等级映射到较大范围的
灰度等级(注:小凸大凹) (整体变暗,y>1)或将图像整体变亮(y<1凸显脊柱改善欠曝光);
由于显示器、打印机等对不同亮度的响应非线性,而是指数$s=r^y$,所以使用
\textbf{y校正}$s=r^\frac{1}{y}$来处理;灰度切片(变换函数T为分段函数)类似于
阈值变换,对于突出图像中某些特征起作用。\\
\textbf{比特平面分层}中高阶比特平面包含了最重要的视觉数据,低阶比特平面贡献了
更精细的灰度细节,存储4个高阶比特平面即可重建原图像(在可接受范围内)。图像相减
时存在-255~255的灰度等级,所以需归一化或者加255除以2将其重新变为0~255,该方法
可用于检测运动的物体或者进行change detection;同一场景多张图取平均噪声水平不变。\\
3.\textbf{直方图均衡化:}(满足单调递增和区间条件),可以使输出图像的灰度分布
更加均衡,提高对比度。(均衡化过程较为简单)直方图:显示了像素值等级分布情况。
$R_r(w)$是概率分布函数。\\
$s=T\left(r\right)=(L-1)\int_{0}^{r}{P_r\left(\omega\right)d\omega=(L-1)\sum_{j=0}^{k}\frac{n_j}{n}}$\\
4.\textbf{平滑线性滤波器:}(均值滤波器)可用于滤除噪声(也可以滤除不必要的细节)或者
突出总体特征,但是会模糊边缘.\textbf{统计排序滤波器:}(中值滤波器)有时表现更佳
(相对average),尤其是滤除椒盐噪声。\textbf{自适应中值滤波器(补)可以滤除空间
密度更大的椒盐噪声,平滑其他噪声并减小失真,没有边缘效应}。自适应中值滤波:
if$\ z_{med}-\ z_{min}>0,\ z_{med}\ {-z}_{max}<0$,\\
$\{if\ \ z_{xy}-\ z_{min}>0,\ z_{xy}-\ z_{max}<0$\\ 
$\ z_{out}=z_{xy}\ else\ z_{out}=z_{med}\}$\ else\ 扩大所取集合的size\\
5.\textbf{锐化空间滤波器:}(减少模糊部分并突出边缘)其效果与平滑空间滤波器
相反(积分与微分之区别)一阶微分$f\left(x+1\right)-f(x)$,二阶微分
$f\left(x+1\right)+f(x-1)-2f(x)$.
一阶微分非0值存在于step和ramp的起点以及ramp沿线;二阶微分非0值存在
于step和ramp的起(终)点;一阶微分产生较粗的边缘,对gray level step有更佳的响应;
二阶微分对细节(细线、孤立点和噪声)有更佳的响应,而对gray level step有双响应
(双边缘,更加明显),因此二阶微分在增强细节方面更强。\\
6.使用\textbf{拉普拉斯算子}锐化图像(中心-4,十字1)。考虑对角项,则掩模系数
变为(中心-8,外圈1)。使用拉普拉斯算子得到的并不是最终图像,还要根据中心系数的
正负,用原图像±拉普拉斯图像。\textbf{最终(中心5,十字-1)(中心9,外圈-1)}\\
7.\textbf{高提升滤波:}$g(x,y)=Af(x,y)\pm\nabla^2f(x,y)$,当A越大,
则越忽略锐化的作用。\\
8.\textbf{使用梯度(一阶微分)锐化图像:}我们直接给出
$g_x=\frac{\partial f}{\partial x}$,$g_y=\frac{\partial f}{\partial y}$这两个模板
称为Sobel算子,此时的操作为线性操作。
再对其求向量$\nabla f$的幅值$M(x,y)=\sqrt{{g_x}^2+{g_y}^2}$或者
近似化$M(x,y)=|gx|+|gy|$,为非线性。\\
Sobel算子X:-1 0 1;-2 0 2;-1 0 1 Y:-1 -2 -1;0 0 0; 1 2 1;\\
Roberts交叉梯度算子X:0 -1;1 0 Y:-1 0;0 1;\\


\end{minipage}%
\hfill
\begin{minipage}[t]{0.315\linewidth}
\textbf{第四章:频率域滤波:}\\
1.图像$f(x,y)$乘以指数项$(-1)^{x+y}$再做傅里叶变换,可将DFT的原点移到
$F(u-\frac{M}{2},v-\frac{N}{2})$不影响幅度谱。若空间域和频域都用极坐标表示,
空间域与频域的旋转等价。$F(0,0))$=图像的平均灰度。\\
2.DFT的幅度谱中,\textbf{低频分量}(幅度谱原点周围)反映了灰度变化缓慢的区域,
是图像在平滑区域上的外观。\textbf{高频分量}反映了灰度剧变的区域(如边缘噪声),
是图像中精细部分。\\
3.\textbf{一般滤波过程:}1.原图乘以${(-1)}^{x+y}$ 2.DFT得到$F(u,v)$
3.$F(u,v)$乘以滤波器$H(u,v)$ 4.IDFT之后再乘以${(-1)}^{x+y}$得到输出。\\
4.\textbf{陷波滤波器}:
\end{minipage}

\newpage

\begin{minipage}[t]{0.315\linewidth}
新的第一列内容
\end{minipage}%
\hfill
\begin{minipage}[t]{0.315\linewidth}
新的第二列内容
\end{minipage}%
\hfill
\begin{minipage}[t]{0.315\linewidth}
新的第三列内容
\end{minipage}



\end{document}