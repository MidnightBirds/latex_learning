\documentclass{ctexart}
\usepackage[landscape, margin=0.5cm]{geometry}
\usepackage{amssymb}

\pagestyle{empty}
\setlength{\parindent}{0pt}
\begin{document}
\fontsize{8.5pt}{10pt}\selectfont
\noindent
\begin{minipage}[t]{0.315\linewidth}
\textbf{第一章绪论:}\\
1.像素在数字图像上具有固定的(x,y)空域坐标以及该点的值f(x,y),其特征为:
x,y,f(x,y)均为离散值和有限值。(f(x,y)常代表图像的intensity或gray等级)
(空域坐标系下y右x)\\
2.数字图像的形式:单通道(B\&W或者灰度等级)\ 三通道(RGB)\ 四通道(RGB+Alpha)深度\\
3. \textbf{(光杆颜锥)}锥状体(少)集中在中央凹附近,对颜色极其敏感;杆状体(多)的分布较为分散,
对低亮度的光照敏感。(两者均不在盲点上有分布)(锥状体(cones),杆状体(rods))\\
4.\textbf{图像不仅仅由电磁波生成,超声波等都可以}(B超)\\
5.\textbf{DIP与计算机视觉、机器人视觉、机器视觉的区别:}\textbf{DIP:}对图像的基础
操作,如几何变换,增强恢复,锐化滤波等,结果仍为图像。\textbf{计算机视觉:}对给定的图
像进行信息提取,如运动检测物体识别,获得具体数据。\textbf{机器视觉:}用于自动化生产,
由机器自主获取图像等信息来获得具体数据,需要多传感器结合。\textbf{机器人视觉:}对象是
机器人,使机器人具备视觉能力完成各项任务。\\
\textbf{第二章数字图像基础:}\\
1.采样量化:数字图像的生成需要经过采样和量化的过程。坐标值(x,y)的数字化就是采样,
值f(x,y)的数字化就是量化。\\
2.\textbf{灰度分辨率:}$2^k$.大小为$M\times N$的数字图像一般用矩阵表示,其占用的储存空间
为:$M\times N\times $k(比特)。空域分辨率为$M\times N$。\\
3.图片的Zooming和Shrinking可用最邻近插值或者双线性插值,前者的缺点是精度低,可能会存在灰度
上的不连续,在变化的地方出现明显的锯齿状。
双线性:$g\left(E\right)=\left(x^\prime-i\right)\left[g\left(B\right)-g\left(A\right)\right]+g\left(A\right)\ $
\\$g\left(F\right)=\left(x^\prime-i\right)\left[g\left(D\right)-g\left(C\right)\right]+g\left(C\right)\ $
\\$g\left(E\right)=\left(x^\prime,y^\prime\right)\left[g\left(F\right)-g\left(E\right)\right]+g(E) $
\\(无论上、降采样,图片所包含内容不变,只是空间分辨率变了)\\
4.像素p的4邻域(十字形$N_4$)、D邻域(4个对角D)和8邻域(一圈$N_8$)。
\textbf{两个像素p和q之间的4邻接(p在q的$N_4$)、8邻接(p在q的$N_8$)和m邻接
(p在q的$N_D$且p的$N_4$与q的$N_4$相交为空;或p在q的$N_4$)}\\
连通(所有连通的像素构成通路):邻接+像素值都属于集合V。
连通分量: 对于S中的任何像素p,S中连通到该像素的像素集
称为S的连通分量。连通集:只有一个连通分量的集合。\\
5.欧几里得距离(De)、街区距离(D4 $\left|x-s\right|+\left|y-t\right|$)和棋盘距离
(D8 max($\left|x-s\right|,\left|y-t\right|$))的概念。(注:D4为菱形,D8为方形)\\
\textbf{第三章:灰度变换和空间滤波(图像增强)高光\ 降噪\ 视觉感染力}\\
1.对于空域,我们的操作一般是针对像素的邻居。若直接对像素本身进行操作,则称为灰度
变换s=T(r)。否则为空间滤波g(x,y)=T(f(x,y))\ ,多使用掩模。\\
2.如果想将一个物体从背景分离,可以使用阈值变换
$s=1.0(r>threshold)+0.0(r<=threshold)$。若输入的图像灰度分辨率很大,可使用
\textbf{对数变换}$s=c\ast log(l+r)$。
\end{minipage}%
\hfill
\begin{minipage}[t]{0.315\linewidth}
\textbf{幂律变化}$s=c\ast r^y$则可以将一个较窄范围的灰度等级映射到较大范围的
灰度等级(注:小凸大凹) (整体变暗,y>1)或将图像整体变亮(y<1凸显脊柱改善欠曝光);
由于显示器、打印机等对不同亮度的响应非线性,而是指数$s=r^y$,所以使用
\textbf{y校正}$s=r^\frac{1}{y}$来处理;灰度切片(变换函数T为分段函数)类似于
阈值变换,对于突出图像中某些特征起作用。\\
\textbf{比特平面分层}中高阶比特平面包含了最重要的视觉数据,低阶比特平面贡献了
更精细的灰度细节,存储4个高阶比特平面即可重建原图像(在可接受范围内)。图像相减
时存在-255~255的灰度等级,所以需归一化或者加255除以2将其重新变为0~255,该方法
可用于检测运动的物体或者进行change detection;同一场景多张图取平均噪声水平不变。\\
3.\textbf{直方图均衡化:}(满足单调递增和区间条件),可以使输出图像的灰度分布
更加均衡,提高对比度。(均衡化过程较为简单)直方图:显示了像素值等级分布情况。
$R_r(w)$是概率分布函数。\\
$s=T\left(r\right)=(L-1)\int_{0}^{r}{P_r\left(\omega\right)d\omega=(L-1)\sum_{j=0}^{k}\frac{n_j}{n}}$\\
4.\textbf{平滑线性滤波器:}(均值滤波器)可用于滤除噪声(也可以滤除不必要的细节)或者
突出总体特征,但是会模糊边缘.\textbf{统计排序滤波器:}(中值滤波器)有时表现更佳
(相对average),尤其是滤除椒盐噪声。\textbf{自适应中值滤波器(补)可以滤除空间
密度更大的椒盐噪声,平滑其他噪声并减小失真,没有边缘效应}。自适应中值滤波:
if$\ z_{med}-\ z_{min}>0,\ z_{med}\ {-z}_{max}<0$,\\
$\{if\ \ z_{xy}-\ z_{min}>0,\ z_{xy}-\ z_{max}<0$\\ 
$\ z_{out}=z_{xy}\ else\ z_{out}=z_{med}\}$\ else\ 扩大所取集合的size\\
5.\textbf{锐化空间滤波器:}(减少模糊部分并突出边缘)其效果与平滑空间滤波器
相反(积分与微分之区别)一阶微分$f\left(x+1\right)-f(x)$,二阶微分
$f\left(x+1\right)+f(x-1)-2f(x)$.
一阶微分非0值存在于step和ramp的起点以及ramp沿线;二阶微分非0值存在
于step和ramp的起(终)点;一阶微分产生较粗的边缘,对gray level step有更佳的响应;
二阶微分对细节(细线、孤立点和噪声)有更佳的响应,而对gray level step有双响应
(双边缘,更加明显),因此二阶微分在增强细节方面更强。\\
6.使用\textbf{拉普拉斯算子}锐化图像(中心-4,十字1)。考虑对角项,则掩模系数
变为(中心-8,外圈1)。使用拉普拉斯算子得到的并不是最终图像,还要根据中心系数的
正负,用原图像±拉普拉斯图像。\textbf{最终(中心5,十字-1)(中心9,外圈-1)}\\
7.\textbf{高提升滤波:}$g(x,y)=Af(x,y)\pm\nabla^2f(x,y)$,当A越大,
则越忽略锐化的作用。\\
8.\textbf{使用梯度(一阶微分)锐化图像:}我们直接给出
$g_x=\frac{\partial f}{\partial x}$,$g_y=\frac{\partial f}{\partial y}$这两个模板
称为Sobel算子,此时的操作为线性操作。
再对其求向量$\nabla f$的幅值$M(x,y)=\sqrt{{g_x}^2+{g_y}^2}$或者
近似化$M(x,y)=|gx|+|gy|$,为非线性。\\
Sobel算子X:-1 0 1;-2 0 2;-1 0 1 Y:-1 -2 -1;0 0 0; 1 2 1;\\
Roberts交叉梯度算子X:0 -1;1 0 Y:-1 0;0 1;\\


\end{minipage}%
\hfill
\begin{minipage}[t]{0.315\linewidth}
\textbf{第四章:频率域滤波:}\\
1.图像$f(x,y)$乘以指数项$(-1)^{x+y}$再做傅里叶变换,可将DFT的原点移到
$F(u-\frac{M}{2},v-\frac{N}{2})$不影响幅度谱。若空间域和频域都用极坐标表示,
空间域与频域的旋转等价。$F(0,0))$=图像的平均灰度。\\
2.DFT的幅度谱中,\textbf{低频分量}(幅度谱原点周围)反映了灰度变化缓慢的区域,
是图像在平滑区域上的外观。\textbf{高频分量}反映了灰度剧变的区域(如边缘噪声),
是图像中精细部分。\\
3.\textbf{一般滤波过程:}1.原图乘以${(-1)}^{x+y}$ 2.DFT得到$F(u,v)$
3.$F(u,v)$乘以滤波器$H(u,v)$ 4.IDFT之后再乘以${(-1)}^{x+y}$得到输出。\\
4.\textbf{陷波滤波器:}$H(u,v)=0\ if\ (u,v)=(\frac{M}{2},\frac{N}{2})$,
$else\ H(u,v)=1 $.\ 可以使图像的平均灰度为0,而不影响图像的整体外观和细节。\\
5.\textbf{理想低通滤波器:},$H(u,v) = 1\ if\left[ D(u,v) \right]< D_0,$
$\ else\ H(u,v) = 0$\ 
其中D(u,v)定义为(u,v)与($\frac{M}{2},\frac{N}{2}$)的欧氏距离。
(注:由于H(u,v)的急剧变化会产生\textbf{振铃现象})\\
6.\textbf{高斯低通滤波器:}$H\left(u,v\right)=e^\frac{-D^2(u,v)}{2{D_0}^2}$
作用:连接断裂处,PS人脸图片上的疤痕或者皱纹。\\
7.\textbf{巴特沃斯低通滤波器:}$H(u,v)=\frac{1}{1+\left[\frac{D(u,v)}{D_0}\right]^{2n}}$
截止频率定义为H(u,v)下降为50\%。一阶二阶巴特沃斯滤波器几乎观察不到振铃现象。
越高阶越接近理想低通。\\
8.\textbf{理想高通滤波器:}$H(u,v) = 0\ if\left[ D(u,v) \right]< D_0,$
$\ else\ H(u,v) = 1$\ \\
9.\textbf{巴特沃斯高通滤波器:}$H(u,v)=\frac{1}{1+\left[\frac{D_0}{D(u,v)}\right]^{2n}}$
越高阶越接近理想高通。\\
10.\textbf{高斯高通滤波器:}$H\left(u,v\right)=1-e^\frac{-D^2(u,v)}{2{D_0}^2}$\\
\textbf{额外:}连续$F(u,v)=\int_{-\infty}^{\infty}\int_{-\infty}^{\infty}f(x,y)e^{-j2\pi xu}e^{-j2\pi yv}dxdy$\\
$f(x,y)=\int_{-\infty}^{\infty}\int_{-\infty}^{\infty}F(x,y)e^{j2\pi xu}e^{j2\pi yv}dudv$\\
离散(dft)$F(u,v)=\sum_{x=0}^{M-1}\sum_{y=0}^{N-1}f(x,y)e^{-j2\pi \frac{xu}{M}}e^{-j2\pi \frac{yv}{N}}$\\
$f(x,y)=\frac{1}{MN}\sum_{x=0}^{M-1}\sum_{y=0}^{N-1}F(x,y)e^{j2\pi \frac{xu}{M}}e^{j2\pi \frac{yv}{N}}$

\textbf{第五章:图像恢复}\\
1.空域上:$g(x,y)=f(x,y)\ast h(x,y)+n(x,y)$\\
频域上:$G(u,v)=F(u,v)\cdot H(u,v)+N(u,v)$\\
图像恢复是由g估计原图像f的过程。\\
2.噪声分布:高斯(常见)$p(z)=\frac{e^{-\frac{{(z-u)}^2}{2\sigma^2}}}{\sqrt{2\pi}\sigma}$
瑞利(无人驾驶)、爱尔兰、Gamma、均值、脉冲、椒盐\\
3.傅里叶谱图有8个点说明原图像有横向纵向对角线的周期性噪声$sin(2\pi u_0x+2\pi v_0y)
\leftrightarrow j\frac{1}{2}[\delta(u+u_0,v+v_0)-\delta(u-u_0,v-v_0)]$\\
4.\textbf{带阻滤波器:}$H\left(u,v\right)=\frac{1}{1+{(\frac{D(u,v)W}{{D(u,v)}^2-{D_0}^2})}^{2n}}$
$H(u,v)=1-e^{-0.5[\frac{D^2(u,v)-D_0^2}{D(u,v)W}]^2}$
\end{minipage}

\newpage

\begin{minipage}[t]{0.315\linewidth}
5.\textbf{陷波滤波器:}notch reject:$H(u,v)=\frac{1}{1+{(\frac{{D0}^2}{D1(u,v)D2(u,v)})}^n}$\\
$H(u,v)=1-e^{-0.5[\frac{D1(u,v) D2(u,v)}{D0^2}]}$ 用以去除噪点:比如说横向纵向噪声\\
6.\textbf{退化模型:}$H(u,v)=e^{-k{(u^2+v^2)}^{5/6}}$\\
7.\textbf{逆滤波:}$\hat{F}(u,v)=\frac{G(u,v)}{H(u,v)}=F(u,v)+\frac{N(u,v)}{H(u,v)}$
则即使我们直到退化过程,由于噪声影响也无法复原。同时必须H不能趋于0,否则会放大噪声。
解决办法:只取H原点附近的频谱,以外一概不考虑,一般来说能量集中于原点,频域幅度在这较高。\\
8.\textbf{维纳滤波器:}使用最小均方误差准则设计\\ $\hat{F}=\left[\frac{H^{\ast}(u,v)}
{\left| H(u,v) \right|  ^2 + S_{\eta}(u,v)/S_f(u,v)}\right]G(u,v)$\\
\textbf{第九章:形态学图像处理(对于二值图)}\\
1.平移:$ (B)_z =\left\{c|c=b+z,b\in B \right\}$(如平移到$(z_1,z_2)$)\\
反射:$ \hat{B} =\left\{w|w=-b,b\in B \right\}$(如$(-x,-y)$反射到$(x,y)$)\\
Fit:完全正确。Hit:部分正确。Miss:全部错误\\
2.\textbf{膨胀(Dilation):}$A\bigoplus B=\left\{ z|(\hat{B})\bigcap A\neq\emptyset \right\}$
或者定义为A与B的镜像存在Hit关系;膨胀会粗化或者增长物体;膨胀可以修复断裂处,修复表面坑洼。
\textbf{(注:膨胀不一定能粗化,和结构元有关。)}\\
3.\textbf{腐蚀(Erosion):}$A\ominus B= \left\{ z|B_z\subseteq A \right\}$
或者定义为A与B存在Fit关系;腐蚀缩小或细化了物体,可视为形态学滤波操作,小于结构元的物体都将滤除。
腐蚀可以分开已连接的物体,可以将物体表面的突出部分剥离。\\
4.\textbf{开运算:}$A\circ B=(A\ominus  B)\oplus B$即对物体先腐蚀再膨胀,其几何解释为:
球形结构元B沿物体A的内部边界滑动,其并集即为开操作
$A\circ B=\bigcup{(B_Z)|(B_Z)\subseteq A}$可以平滑图像轮廓,打断物体间的连接部分,清除突出物。\\
5.\textbf{闭运算:}$A\bullet B=(A\circ B)\ominus B$即对物体先膨胀再腐蚀,其几何解释为:
球形结构元B沿物体A的边界滑动,其并集即为闭操作
$A\bullet B=\bigcap{(B_Z)|(B_Z)\subseteq A}$可以平滑轮廓上的缺口,填充洞口,连接间隙和断裂部分。\\
6.\textbf{Hit or Miss:}$A \circledast B=\left(A\ominus X\right)\cap[A^C\ominus(W-X)]$
设感兴趣物体形状为X,B为X及其背景组成的集合,W为比X大一点的小窗,用X腐蚀A产生的集合
与用(W-X)腐蚀A的补集产生的集合的交集就是击中或击不中变换。\\
7.\textbf{边缘提取:}$\beta\left(A\right)=A-(A\ominus B)$即可提取边界,B是结构元。\\
8.\textbf{区域填充:}需设置一个初始点,过程可表示为\\
$X_k=(X_{k-1}\oplus B)\cap A^C$,直到$X_k=X_{k-1}$,完成区域填充,最终结果图应为$X_k\cup A$\\
9.连通分量的提取:过程与上面类似,只是将$A^C$替换成A,最终结果图应为$X_k$。\\
\textbf{第十章:图像分割(基于不连续性与相似性)}\\
1.灰度变化的不连续性是进行图像分割的基础之一,其处理的图像特征为\textbf{孤立点、线和边缘。}
对于相似的灰度,根据一组预定义的准则将图像分割为相似的区域。
\textbf{(阈值处理,区域生长、分裂和聚合)}\\




\end{minipage}%
\hfill
\begin{minipage}[t]{0.315\linewidth}
2.\textbf{点检测:}使用二阶微分(实际就是进行空间滤波)如:拉普拉斯算子(添加对角项)
的模板系数之和为0,表明在恒定灰度区域的模板响应为0。设置一个阈值T,若模板响应$R(x,y)>=T$,
则输出图像在该点的值$g(x,y)=1$,检测到孤立点。\\
3.\textbf{线检测:}同样可用拉普拉斯算子。由于其产生负值,我们一般进行正阈值处理,
仅适用拉普拉斯图像的正值。(注:当线宽比模板尺寸大时,会被一个零值分开)
。若我们对特定方向的线感兴趣,可将模板上对应方向的系数全换成2
(最后同样要进行阈值处理)\\
4.\textbf{Ramp模型:}线的厚度与斜率成反比,斜率与模糊程度成反比。\\
5.微弱的可见噪声也严重影响边缘检测所用的两个关键导数,所以应先进行平滑处理。
另一方法是对梯度图像进行阈值处理(可能会使部分边缘断开)。若目的是突出主要边缘并尽可能保持连接时,
实践中通常又做平滑处理又做阈值处理。\\
6.
11.\textbf{区域分割:}(注:二维可用四叉树 三维用八叉树)
\end{minipage}%
\hfill
\begin{minipage}[t]{0.315\linewidth}
9.\textbf{Ostu方法(最佳全局阈值处理)}:计算快速简单,不受图像亮度和对比度影响。
但是对噪声敏感,而且只能对单一目标分割;当目标和背景大小相差悬殊时,效果不好。\\
1\textbf{品红$M=R+B$,青色$C=G+B$,黄色$Y=R+G$}。\\
2.全彩色图像有24比特的深度,RGB分别为8比特图像。(RGB归一化)
\textbf{注:RGB不能表征所有颜色 HIS比RGB更易表征颜色。}\\
3. CMY 模型(面向应用)\textbf{注:CIE国际标准来看显示器和打印机能实现颜色不一样}.\\
4.\textbf{HSI模型:H为色调,S为饱和度,I为强度}:对锥体模型:离原点在z轴方向的距离
表示I,当前圆面内用极坐标表示,离圆心的距离r表征S,夹角$\theta$表示H。\\
5.\textbf{RGB到HSI:}\\
7.\textbf{灰度分层:}类似于灰度切片,但是将不同分段的像素集合赋予不同的颜色,
有所区别。(伪彩色)
8.\\
9.\textbf{对彩色图像的滤波:}在HIS模型一般只能对I处理\\
11.\textbf{彩色轮廓提取:} 1.RGB转HSI\ 2.取对应颜色的his,在对应阈值内置白色,
阈值外置黑色\ 3.滤波\ 4.findContours寻找白色区域轮廓\ 5.drawContours绘制轮廓。
\end{minipage}



\end{document}